%package obligatoire : type de document
\documentclass[a4paper,12pt,twoside]{book}

% encodage
\usepackage{fontspec}

% le package hyperref avec des options, si en local
\usepackage[pdfusetitle, pdfsubject ={Mémoire TNAH}, pdfkeywords={les mots-clés}]{hyperref}

%il faut mettre au moins une langue
\usepackage[english,french]{babel}

% configurer le document selon les normes de l'école
\usepackage[margin=2.5cm]{geometry} %marges
\usepackage{setspace} % espacement qui permet ensuite de définir un interligne
\onehalfspacing % interligne de 1.5
\setlength\parindent{1cm} % indentation des paragraphes à 1 cm

% Table des matières
\addto\captionsfrench{
\renewcommand*\contentsname{Contenu de la documentation}
}

% bibliographie
\usepackage[backend=biber, sorting=nyt, style=enc,maxbibnames=10]{biblatex}
\addbibresource{biblio.bib}
%\nocite{*}

% Sigles et acronymes
\usepackage[automake,acronym,toc]{glossaries}
\makeglossaries
\newacronym{cds}{CdS}{Constance de Salm}


% DOCUMENT
\begin{document}
	
	\tableofcontents
	
	\chapter{HTR}
		
		\section{Problématique}
		Quatre à cinq mains différentes ont été repérées jusqu'à présent dans la correspondance de \gls{cds}. Cette variété d'écritures peut sérieusement entraver les performances d'un modèle de reconnaissance.
		
		Deux pistes méthodologiques se dessinent :
		\begin{enumerate}
			\item Rassembler dans un premier temps des lettres qui sont de la même main, pour voir quels sont les résultats du modèle qu'H. Souvay a commencé à entraîner ;
			\item Reprendre un modèle déjà entraîné à travailler sur plusieurs mains ; c'est l'option qui été privilégiée par le projet Lectaurep\footcite{chagueCreationModelesTranscriptiona}).
		\end{enumerate}
		
				
		\section{Choisir un corpus d'entraînement}
		Les recueils de lettres constituent la part du corpus la plus normée sur le plan paléographique. La distribution des mains y est variable selon les tomes :
		\begin{enumerate}
			\item Le premier volume\footcite{salmCorrespondanceGeneraleSecondea} présente une grande variété de main s'enchaînant fréquemment les unes aux autres ;
			\item Le deuxième volume\footcite{salmCorrespondanceGeneraleSeconde}  présente en revanche une meilleure cohérence paléographique : la même main peut se suivre sur un bon nombre de pages consécutives, facilitant l'entraînement d'un modèle sur une écriture particulière. En partie utilisé par H. Souvay pour ses tests, nous avons repris ce volume pour constituer un premier sous-corpus paléographiquement cohérent.
		\end{enumerate}
	
		\subsection{Main 1}
		Nous avons établi une liste de 47 images (soit 47 doubles pages) au sein du 2e volume attestant une écriture homogène que nous dénommons \textit{Main 1}. Nous avons pour cela arbitrairement découpé les lettres afin de ne travailler que sur un seul type d'écriture, sachant que les changements de main interviennent souvent en milieu de page. Quelques corrections de la main de \gls{cds} apparaissent ponctuellement\footnote{Des reproductions en qualité réduite de ces images ont été placées dans le dossier \textsf{./htr/img/main1bd}}.
		
		\subsection{Écriture de \gls{cds}}
		Le site ne publie aucune lettre originale de la main de \gls{cds}, mais 52 brouillons (\textit{Entwurf})\footnote{Le dépouillement se trouve dans le fichier \textsf{./htr/mains/brouillonsCDS.md}}.
		
		Entraîner un modèle de reconnaissance sur cette écriture supposerait un travail délicat de transcription pour une écriture particulièrement cursive (compter environ deux semaines pour disposer d'une bonne vingtaine de pages), mais l'investissement peut en valoir la peine.
		
		\section{Segmentation et typage des zones d'écriture}
		Nous 
				
\end{document}